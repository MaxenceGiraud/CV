%----------------------------------------------------------------------------------------
% Medium Length Professional CV
% LaTeX Template
% Version 2.0 (8/5/13)
%
% This template has been downloaded from:
% http://www.LaTeXTemplates.com
%
% Original author:
% Rishi Shah 
%
% Important note:
% This template requires the resume.cls file to be in the same directory as the
% .tex file. The resume.cls file provides the resume style used for structuring the
% document.
%
%----------------------------------------------------------------------------------------

%----------------------------------------------------------------------------------------
%	PACKAGES AND OTHER DOCUMENT CONFIGURATIONS
%----------------------------------------------------------------------------------------

\documentclass{resume} % Use the custom resume.cls style
\usepackage[left=0.8in,top=0.6in,right=0.8in,bottom=0.6in]{geometry} % Document margins

\usepackage[french, english]{babel} % multi-language support
\usepackage{iflang}
\usepackage{hyperref}
\usepackage{graphicx}

\newcommand{\translation}[2]{\IfLanguageName{#1}{#2}{}}
\newcommand{\en}[1]{\translation{english}{#1}}
\newcommand{\fr}[1]{\translation{french}{#1}}
\newcommand{\lang}{english} % CONTROL OUTPUT LANGUAGE

\newcommand{\tab}[1]{\hspace{.2667\textwidth}\rlap{#1}}
\newcommand{\itab}[1]{\hspace{0em}\rlap{#1}}

\long\def\/*#1*/{} %% allows multiline comments
%\newcommand{\babel}[2]{\IfLanguageName{english}{#1}{#2}}

\name{Maxence Giraud} % Your name
\address{maxence.giraud.etu@univ-lille.fr \\  \href{https://github.com/MaxenceGiraud}{\includegraphics[width=0.35cm]{github-logo.pdf}} } % Your address
\address{ Master 2 Data Science / Machine Learning} % phone number and email

\begin{document}
\expandafter\selectlanguage\expandafter{\lang}

%----------------------------------------------------------------------------------------
%	EDUCATION SECTION
%----------------------------------------------------------------------------------------

\begin{rSection}{Education}

    \begin{rSubsection}{Master DATA Science - Université de Lille}{\em 2019 - 2021}
        {\en{Master Research in partnership with the University of Lille, Centrale Lille and IMT Lille Douai} \fr{Master Recherche en partenariat avec L'Université de Lille, Centrale Lille et IMT Lille Douai}}

        \en{
        \item Machine Learning/Deep Learning, Statistics, Probabilities, Optimization, Signal Processing, Linear Algebra
        \item Sequential Decision Making (MAB, Reinforcement learning), Bayesian learning, graph based ML, Privacy ML 
        }
        
        \fr{
        \item Machine Learning/Deep Learning, Statistiques, Probabilités, Optimisation, traitement du signal, Algèbre linéaire
        \item Cours en anglais}
    \end{rSubsection}


    \begin{rSubsection}{IMT Lille Douai}{\em 2014 - 2019}
        {\en{Engineering School - Computer Science}\fr{Ecole d'ingénieur}}{}
        \item \en{Preparatory classes + Engineering path in Computer Science / Networking / Telecommunication. } \fr{Classe préparatoires intégrées et parcours Ingénieur Informatique / Réseaux/ Telecommunication.}
    \end{rSubsection}

\end{rSection}

%--------------------------------------------------------------------------------
%    Projects And SeminarsFirst approach to academic research.
%-----------------------------------------------------------------------------------------------
\begin{rSection}{\en{Research Projects} \fr{Projets Recherches}}

    {\bf Research Project M2 - University of Lille, team Sigma (Cristal lab) }\hfill {\em 2020-2021}
    \en{
        \\ Tensorization of data in order to compare cluster centers without computing them.
        \\ Implementation in Python+Matlab/ Multilinear algebra / Statistics}
    \fr{
        \\ Tensorization of data in order to compare cluster centers without computing them.
        \\ Implementation in Matlab/ Multilinear algebra / Statistics}

    {\bf Research Project M1 - University of Lille, team Biocomputing (Cristal lab) }\hfill {\em 2019-2020}
    \en{
        \\ Mathematical Modeling of glucose in cells
        \\ Model made in python (using simpy + scipy), using several differential equations.
        \\ First approach to academic research.}
    \fr{
        \\  Modelisation mathématique du glucose dans les cellules epithelial enterocytes
        \\ Modèle développé en python (using simpy + scipy), composé principalement d'équations différentielles.
        \\ Première approche à la recherche académique.}

\end{rSection}
%----------------------------------------------------------------------------------------
%	WORK EXPERIENCE SECTION
%----------------------------------------------------------------------------------------

\begin{rSection}{\en{Work Experience}\fr{Expérience}}


    \begin{rSubsection}{University of Lille - Team Sigma (Cristal lab)}{\em April-\en{July}\fr{Juil} 2020}{\en{Research Intern} \fr{Chercheur Stagiaire}}{}
        \en{
            \item Simulate in real time Battery pack of electrified vehicles using Machine Learning models.
            \item Overview of the state of the arts of ML/DL models (SVM, RandomForest,GradientBoosting,MLP,RNN ...)
            \item Optimization of models, Computing confidence intervals, Studies of the models' generalisation.}
        \fr{
            \item Simuler en temps des batteries de voitures electriques avec des modèles de Machine Learning.
            \item Effectuer un état de l'art des algorithms de ML/DL (SVM,RandomForest, GradientBoosting, MLP, RNN ...).
            \item Optimisation des modèles, Calculs d'intervales de confiance, Test de généralisation des modèles.}
    \end{rSubsection}

    \begin{rSubsection}{Orange Cyberdéfense, Lesquin}{\em Jan-\en{July}\fr{Juil} 2019}{\en{Intern Security Engineer} \fr{Stagiaire Ingénieur Sécurité}}{}
        \en{
            \item Automation of a FortiGate Firewall park deployment using FortiManager's API.
            \item Python Scripts Programming and Networking/ Security work}
        \fr{
            \item Automatisation du déploiement d'un parc de  Firewall Fortinet via l’API de FortiManager.
            \item Programmation de script python, Mise en place de matériel de Réseaux et sécurité.
        }
    \end{rSubsection}

    \begin{rSubsection}{Coreye, Villeneuve D'Ascq}{\em Sept 2018 - Jan 2019}{\en{Intern Developper} \fr{Développeur stagiaire}}{}
        \en{
            \item Automation of Fortinet Firewall deployment using its API in Ruby.
            \item Firewalls Migration, Remplacement of  monitoring networking tools.}
        \fr{
            \item Automatisation du déploiement de Firewall Fortinet via l’API.
            \item Migration de Firewalls. Remplacement des sondes de supervisions.}
    \end{rSubsection}
    

    \begin{rSubsection}{Hochschule Für Telekomunication, Leipzig - Germany}{\em \en{April - July}\fr{Avril - Juil} 2017}{\en{Intern} \fr{Stagiaire}}{}
        \en{
            \item Analysis and comparaison of 3 virtualisation systems:
            Cloudstack / Openstack \\ / Oracle VM server.}
        \fr{
            \item Analyse et comparaison de 3 systèmes de virtualisation : Cloudstack / Openstack / Oracle VM server. Et préconisation d’utilisation en fonction des contextes.
        }

    \end{rSubsection}

    %---------- old internships ----------------------------------------------------------
    % \begin{rSubsection}{Vetoch, Villeneuve D'Ascq - France}{\em Sept 2017- Jan2018 }{\en{ Marketing Intern} \fr{Stagiaire Marketing}}{}
    %     \en{
    %         \item Mise en pratique marketing et communication.
    %         \item Création d’un site Web et d’un Extranet (Wordpress).
    %         \item Traduction de nombreux documents en anglais.}
    %     \fr{
    %         \item Mise en pratique marketing et communication.
    %         \item Création d’un site Web et d’un Extranet (Wordpress).
    %         \item Traduction de nombreux documents en anglais.
    %     }
    % \end{rSubsection}

    % \begin{rSubsection}{Supermarché Match - La Madeleine France}{\em \en{Feb - April}\fr{Février - Avril} 2015}{\en{First experience Internship} \fr{Stage Découverte de l'entreprise}}{}
    %     \en{
    %         \item Assistance aux contrôleurs de gestion.
    %         \item Automatisation de process avec Excel (Développement de
    %         macro).}
    %     \fr{
    %         \item Assistance aux contrôleurs de gestion.
    %         \item Automatisation de process avec Excel (Développement de
    %         macro).
    %     }
    % \end{rSubsection}
    %------------------------------------------------------------------------------------

\end{rSection}

%----------------------------------------------------------------------------------------
%	TECHNICAL STRENGTHS SECTION
%----------------------------------------------------------------------------------------

\begin{rSection}{\en{Technical Strengths} \fr{Compétences techniques}}

    \begin{tabular}{ @{} >{\bfseries}l @{\hspace{6ex}} l }
        \en{Mathematics                                          & Probabilities, Statistics, Linear Algebra \\}
        \fr{Mathématiques                                        & Probabilités, Statistiques, Algèbre linéaire.\\}
        \en{Programming Languagues} \fr{Languages programmation} & Python (Numpy, scikit-learn, tensorflow,scipy), C, R, Ruby                                                     \\
        Machine Learning                                         & Linear regression (LS), LDA/QDA/FDA, SVM, dimension reduction (PCA)    \\ \emph{(\en{Theory and practice}\fr{Théorie et pratique})} & Unsupervised learning (k-means/medoids), Decision trees (Random Forest) \\
                                                                 & Deep Learning (MLP,CNN,RNN,GAN,Autoencoder/VAE)                                  \\
        \en{Other Tools} \fr{Autres Outils}                                   & Latex, Beamer                                                          \\
    \end{tabular}

\end{rSection}

%----------------------------------------------------------------------------------------
%	LANGUAGES SECTION
%----------------------------------------------------------------------------------------

\begin{rSection}{Languages}

    \begin{tabular}{@{} >{\bfseries}l @{\hspace{4ex}} l @{\hspace{6ex}} @{} >{\bfseries}l @{\hspace{4ex}} l @{\hspace{6ex}} @{} >{\bfseries}l @{\hspace{4ex}} l}
        \en{
        French   & Mother Tongue & English  & Bilingual & German  & Intermediate \\}
        \fr{
        Français & Langue maternelle  & Anglais  & Bilingue & Allemand & Intermédaire \\}
    \end{tabular}

\end{rSection}



%----------------------------------------------------------------------------------------
%	Hobbies/Interests SECTION
%----------------------------------------------------------------------------------------

% \begin{rSection}{Hobbies and Other interests}
% Handball, Swimming, Astronomy, Physics.
% \end{rSection}

\/*  %% Commented

*/

\end{document}


